% Isto é um exemplo de Ficha Catalográfica, ou ``Dados internacionais de
% catalogação-na-publicação''. Você pode utilizar este modelo como referência. 
% Porém, provavelmente a biblioteca da sua universidade lhe fornecerá um PDF
% com a ficha catalográfica definitiva após a defesa do trabalho. Quando estiver
% com o documento, salve-o como PDF no diretório do seu projeto e substitua todo
% o conteúdo de implementação deste arquivo pelo comando abaixo:

%\begin{fichacatalografica}
%    ~\\
%    \vfill
%    \begin{center}
%    \large\textbf{FICHA CATALOGRÁFICA} \par
%    \vspace{0.4cm}
%    \hspace{-1cm}
%    \makebox[\textwidth+0.2cm][l]{\includegraphics[]{./PreTextual/FichaCatalografica.pdf}}
%    \end{center}
%    ~\\
%
%\end{fichacatalografica}

\begin{fichacatalografica}
	\vspace*{\fill}					% Posição vertical
	\hrule							% Linha horizontal
	\begin{center}					% Minipage Centralizado
	\begin{minipage}[c]{12.5cm}		% Largura
	
	\imprimirautor
	
	\hspace{0.5cm} \imprimirtitulo  / \imprimirautor. --
	\imprimirlocal, \imprimirdata-
	
	\hspace{0.5cm} \pageref{LastPage} p. : il. (algumas color.) ; 30 cm.\\
	
	\hspace{0.5cm} \imprimirorientadorRotulo~\imprimirorientador\\
	
	\hspace{0.5cm}
	\parbox[t]{\textwidth}{\imprimirtipotrabalho~--~\imprimirinstituicao,
	\imprimirdata.}\\
	
	\hspace{0.5cm}
		1. Análise de trajetórias.
		2. Identificação de perfil.
		I. Prof. Dr. José de Oliveira.
		II. Universidade do Estado de Santa Catarina.
		III. Centro de Ciências Tecnológicas.
		IV. identificação paramétrica em malha fechada do motor de indução trifásico usando o método de mínimos quadrados\\ 			
	
	\hspace{8.75cm} CDU 02:141:005.7\\
	
	\end{minipage}
	\end{center}
	\hrule
\end{fichacatalografica}