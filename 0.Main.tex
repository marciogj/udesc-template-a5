\documentclass[
	% -- opções da classe memoir --
	11pt,				% tamanho da fonte
	openright,			% capítulos começam em pág ímpar (insere página vazia caso preciso)
	twoside,			% para impressão em verso e anverso. Oposto a oneside
	a5paper,			% tamanho do papel. 
	% -- opções da classe abntex2 --
	chapter=TITLE,		% títulos de capítulos convertidos em letras maiúsculas
	%section=TITLE,		% títulos de seções convertidos em letras maiúsculas
	%subsection=TITLE,	% títulos de subseções convertidos em letras maiúsculas
	%subsubsection=TITLE,% títulos de subsubseções convertidos em letras maiúsculas
	% -- opções do pacote babel --
	english,			% idioma adicional para hifenização
	% french,				% idioma adicional para hifenização
	% spanish,			% idioma adicional para hifenização
	brazil				% o último idioma é o principal do documento
	]{abntex2}

\usepackage{setspace}
\usepackage{Template/udesc-dissertacao}
\usepackage{pdfpages}
\usepackage{amsmath}
\usepackage{listings}
\usepackage{minted}
\usepackage{varioref}
\usepackage{tablefootnote}
\usepackage{afterpage}

%\usepackage{algpseudocode}
%\usepackage[options]{algorithm2e}

% Declaracoes em Português
\algrenewcommand\algorithmicend{\textbf{fim}}
\algrenewcommand\algorithmicdo{\textbf{faça}}
\algrenewcommand\algorithmicwhile{\textbf{enquanto}}
\algrenewcommand\algorithmicfor{\textbf{para}}
\algrenewcommand\algorithmicif{\textbf{se}}
\algrenewcommand\algorithmicthen{\textbf{então}}
\algrenewcommand\algorithmicelse{\textbf{senão}}
\algrenewcommand\algorithmicreturn{\textbf{devolve}}
\algrenewcommand\algorithmicfunction{\textbf{função}}

% Rearranja os finais de cada estrutura
\algrenewtext{EndWhile}{\algorithmicend\ \algorithmicwhile}
\algrenewtext{EndFor}{\algorithmicend\ \algorithmicfor}
\algrenewtext{EndIf}{\algorithmicend\ \algorithmicif}
\algrenewtext{EndFunction}{\algorithmicend\ \algorithmicfunction}

% O comando For, a seguir, retorna 'para #1 -- #2 até #3 faça'
%\algnewcommand\algorithmicto{\textbf{até}}
%\algrenewtext{For}[3]%
%{\algorithmicfor\ #1 $\gets$ #2 \algorithmicto\ #3 \algorithmicdo}

\renewcommand{\UrlBreaks}{\do\/\do\a\do\b\do\c\do\d\do\e\do\f\do\g\do\h\do\i\do\j\do\k\do\l\do\m\do\n\do\o\do\p\do\q\do\r\do\s\do\t\do\u\do\v\do\w\do\x\do\y\do\z\do\A\do\B\do\C\do\D\do\E\do\F\do\G\do\H\do\I\do\J\do\K\do\L\do\M\do\N\do\O\do\P\do\Q\do\R\do\S\do\T\do\U\do\V\do\W\do\X\do\Y\do\Z}

% para remover a hifenização
%\hyphenpenalty=10000
%\exhyphenpenalty=10000


% ---
% Informações de dados para CAPA e FOLHA DE ROSTO
% ---
\titulo{TITULO}
\autor{AUTOR}
\local{Joinville}
\newcommand{\estado}{SC}
\data{2016}
\orientador{ORIENTADOR}
\coorientador{}
\instituicao{
  Universidade do Estado de Santa Catarina - UDESC
  \par
  Centro de Ciências Tecnológicas - CCT
  \par
  Programa de Pós-Graduação em Computação Aplicada - PPGCA}
\tipotrabalho{Dissertação (Mestrado)}
% O preambulo deve conter o tipo do trabalho, o objetivo, 
% o nome da instituição e a área de concentração 
\preambulo{Dissertação apresentada ao Programa de Pós-Graduação em Computação Aplicada da Universidade do Estado de Santa Catarina, como requisito parcial para obtenção do grau de Mestre em Computação Aplicada.}
% ---

% ---
% Configurações de aparência do PDF final
% informações do PDF
\makeatletter
\hypersetup{
     	%pagebackref=true,
		pdftitle={\@title}, 
		pdfauthor={\@author},
    	pdfsubject={\imprimirpreambulo},
	    pdfcreator={pdflatex},
		pdfkeywords={abnt}{latex}{abntex}{abntex2}{trabalho acadêmico}, 
		colorlinks=true,       		% false: boxed links; true: colored links
    	linkcolor=blue,          	% color of internal links
    	citecolor=blue,        		% color of links to bibliography
    	filecolor=magenta,      		% color of file links
		urlcolor=blue,
		bookmarksdepth=4
}
\makeatother
% --- 

% ---
% compila o indice
% ---
\makeindex
% ---

% ----
% Início do documento
% ----
\begin{document}

% Retira espaço extra obsoleto entre as frases.
\frenchspacing 


% \pretextual

% ---
% Capa
% ---
%\imprimircapa
% ---

% ---
% Folha de rosto
% (o * indica que haverá a ficha bibliográfica)
% ---
\imprimirfolhaderosto*
% ---

% Isto é um exemplo de Ficha Catalográfica, ou ``Dados internacionais de
% catalogação-na-publicação''. Você pode utilizar este modelo como referência. 
% Porém, provavelmente a biblioteca da sua universidade lhe fornecerá um PDF
% com a ficha catalográfica definitiva após a defesa do trabalho. Quando estiver
% com o documento, salve-o como PDF no diretório do seu projeto e substitua todo
% o conteúdo de implementação deste arquivo pelo comando abaixo:

%\begin{fichacatalografica}
%    ~\\
%    \vfill
%    \begin{center}
%    \large\textbf{FICHA CATALOGRÁFICA} \par
%    \vspace{0.4cm}
%    \hspace{-1cm}
%    \makebox[\textwidth+0.2cm][l]{\includegraphics[]{./PreTextual/FichaCatalografica.pdf}}
%    \end{center}
%    ~\\
%
%\end{fichacatalografica}

\begin{fichacatalografica}
	\vspace*{\fill}					% Posição vertical
	\hrule							% Linha horizontal
	\begin{center}					% Minipage Centralizado
	\begin{minipage}[c]{12.5cm}		% Largura
	
	\imprimirautor
	
	\hspace{0.5cm} \imprimirtitulo  / \imprimirautor. --
	\imprimirlocal, \imprimirdata-
	
	\hspace{0.5cm} \pageref{LastPage} p. : il. (algumas color.) ; 30 cm.\\
	
	\hspace{0.5cm} \imprimirorientadorRotulo~\imprimirorientador\\
	
	\hspace{0.5cm}
	\parbox[t]{\textwidth}{\imprimirtipotrabalho~--~\imprimirinstituicao,
	\imprimirdata.}\\
	
	\hspace{0.5cm}
		1. Análise de trajetórias.
		2. Identificação de perfil.
		I. Prof. Dr. José de Oliveira.
		II. Universidade do Estado de Santa Catarina.
		III. Centro de Ciências Tecnológicas.
		IV. identificação paramétrica em malha fechada do motor de indução trifásico usando o método de mínimos quadrados\\ 			
	
	\hspace{8.75cm} CDU 02:141:005.7\\
	
	\end{minipage}
	\end{center}
	\hrule
\end{fichacatalografica}
%% ---
% Errata
% ---
% \begin{errata}
% Elemento opcional da \citeonline[4.2.1.2]{NBR14724:2011}. Exemplo:
% 
% \vspace{\onelineskip}
% 
% FERRIGNO, C. R. A. \textbf{Tratamento de neoplasias ósseas apendiculares com
% reimplantação de enxerto ósseo autólogo autoclavado associado ao plasma
% rico em plaquetas}: estudo crítico na cirurgia de preservação de membro em
% cães. 2011. 128 f. Tese (Livre-Docência) - Faculdade de Medicina Veterinária e
% Zootecnia, Universidade de São Paulo, São Paulo, 2011.
% 
% \begin{table}[htb]
% \center
% \footnotesize
% \begin{tabular}{|p{1.4cm}|p{1cm}|p{3cm}|p{3cm}|}
%   \hline
%    \textbf{Folha} & \textbf{Linha}  & \textbf{Onde se lê}  & \textbf{Leia-se}  \\
%     \hline
%     1 & 10 & auto-conclavo & autoconclavo\\
%    \hline
% \end{tabular}
% \end{table}
% 
% \end{errata}
% ---
% Isto é um exemplo de Folha de aprovação, elemento obrigatório da NBR
% 14724/2011 (seção 4.2.1.3). Você pode utilizar este modelo até a aprovação
% do trabalho. Após isso, substitua todo o conteúdo deste arquivo por uma
% imagem da página assinada pela banca com o comando abaixo:
%

%\includepdf{./PreTextual/Assinaturas.pdf}


\begin{folhadeaprovacao}
    \begin{center}
    \vspace*{\fill}
  	{\ABNTEXchapterfont\bfseries\Large\MakeUppercase{\imprimirautor}}
  	
  	\vspace*{\fill}
  	
    {\ABNTEXchapterfont\bfseries\Large\imprimirtitulo}
    \end{center}
    
    \vspace*{\fill}
    Dissertação apresentada ao Curso de Mestrado acadêmico Computação Aplicada como requisito parcial para obtenção do título de Mestre em Computação Aplicada na área de concentração ``Ciência da Computação''
    
    \begin{flushleft}
	\textbf{Banca Examinadora:}
	\end{flushleft}
	
    \begin{center}
	\vspace*{\fill}
	\begin{minipage}[h!]{0.35\textwidth}
	\centering
	\vspace*{-6cm}
	\begin{flushleft}
	Orientador:
	\end{flushleft}
	
	\vspace*{1cm}
	\begin{flushleft}
	\textbf{Membros:}
	\end{flushleft}

	\end{minipage}    
   	\hfill
	\begin{minipage}[h!]{0.62\textwidth}
   \assinatura{\imprimirorientador \\ CCT/UDESC} 
   \assinatura{Professor1 \\ DPTO/UNIVERSIDADE}
   \assinatura{Professor2 \\ DPTO/UNIVERSIDADE}
   \assinatura{Professor3 \\ DPTO/UNIVERSIDADE}
   %\assinatura{\textbf{Professor} \\ Convidado 4}
   \end{minipage}  
   

    \vspace*{1cm}
    {\large{\textbf{\imprimirlocal , \estado, \textbf{\imprimirdata}.}}}
   \vspace*{0.1cm}
   
   \end{center}
\end{folhadeaprovacao}

%falta uma pagina em branco
%\newpage\null\thispagestyle{empty}\newpage

\begin{dedicatoria}
   \vspace*{\fill}
   \centering
   \noindent
   \textit{Dedico esse trabalho a,\\
   lorem ipsum.} \vspace*{\fill}
\end{dedicatoria}
% ---


\begin{agradecimentos}
Lorem ipsum dolor sit amet, consectetur adipiscing elit. Phasellus bibendum erat vel dapibus finibus. Integer dignissim feugiat dui eget semper. Integer in lacinia mi. Nunc pellentesque accumsan ante quis egestas. Proin accumsan laoreet pellentesque. Nunc varius, orci quis maximus faucibus, nunc eros finibus lorem, vel aliquam lacus dolor ut lorem. Donec a justo in leo pretium pellentesque. Praesent a efficitur nunc. Donec pulvinar diam odio, ac congue magna bibendum vehicula. Nunc sed erat non est pulvinar sollicitudin mattis eget sem. Nulla laoreet egestas mattis. Donec eu fermentum mi. Nam dolor metus, convallis vel ligula eu, hendrerit scelerisque enim. Nunc quis dui id eros ultricies scelerisque ac nec neque.

\end{agradecimentos}
\begin{epigrafe}
    \vspace*{\fill}
	\begin{flushright}
		\textit{``Se eu vi mais longe,\\
foi por estar sobre ombros de gigantes.''\\
        (Isaac Newton)}
	\end{flushright}
\end{epigrafe}
% resumo em português
\setlength{\absparsep}{18pt} % ajusta o espaçamento dos parágrafos do resumo

\begin{resumo}
SOBRENOME, NOME. \textbf{TITULO}. \imprimirdata. \pageref{LastPage} p. Dissertação (Mestrado em Computação Aplicada - Área: Engenharia (\textit{Engineering})). Universidade do Estado de Santa Catarina. Programa de Pós-Gradua-ção em Computação Aplicada. \imprimirlocal, \imprimirdata.\\
\\
Lorem ipsum dolor sit amet, consectetur adipiscing elit. Phasellus bibendum erat vel dapibus finibus. Integer dignissim feugiat dui eget semper. Integer in lacinia mi. Nunc pellentesque accumsan ante quis egestas. Proin accumsan laoreet pellentesque. Nunc varius, orci quis maximus faucibus, nunc eros finibus lorem, vel aliquam lacus dolor ut lorem. Donec a justo in leo pretium pellentesque. Praesent a efficitur nunc. Donec pulvinar diam odio, ac congue magna bibendum vehicula. Nunc sed erat non est pulvinar sollicitudin mattis eget sem. Nulla laoreet egestas mattis. Donec eu fermentum mi. Nam dolor metus, convallis vel ligula eu, hendrerit scelerisque enim. Nunc quis dui id eros ultricies scelerisque ac nec neque.

Maecenas vestibulum ligula eu nibh ornare, quis placerat massa iaculis. In et nunc malesuada, elementum justo vitae, iaculis urna. Cras pharetra metus et velit aliquam, porttitor pharetra turpis varius. Lorem ipsum dolor sit amet, consectetur adipiscing elit. Duis consectetur tincidunt elementum. Interdum et malesuada fames ac ante ipsum primis in faucibus. Ut leo diam, viverra ultrices neque lobortis, eleifend viverra erat. 
\vspace{\onelineskip}
 
\noindent 
   
\textbf{Palavras-chaves}: palavra1, palavra2.
\end{resumo}
\begin{resumo}[Abstract]
\begin{otherlanguage*}{english}
Lorem ipsum dolor sit amet, consectetur adipiscing elit. Phasellus bibendum erat vel dapibus finibus. Integer dignissim feugiat dui eget semper. Integer in lacinia mi. Nunc pellentesque accumsan ante quis egestas. Proin accumsan laoreet pellentesque. Nunc varius, orci quis maximus faucibus, nunc eros finibus lorem, vel aliquam lacus dolor ut lorem. Donec a justo in leo pretium pellentesque. Praesent a efficitur nunc. Donec pulvinar diam odio, ac congue magna bibendum vehicula. Nunc sed erat non est pulvinar sollicitudin mattis eget sem. Nulla laoreet egestas mattis. Donec eu fermentum mi. Nam dolor metus, convallis vel ligula eu, hendrerit scelerisque enim. Nunc quis dui id eros ultricies scelerisque ac nec neque.

Maecenas vestibulum ligula eu nibh ornare, quis placerat massa iaculis. In et nunc malesuada, elementum justo vitae, iaculis urna. Cras pharetra metus et velit aliquam, porttitor pharetra turpis varius. Lorem ipsum dolor sit amet, consectetur adipiscing elit. Duis consectetur tincidunt elementum. Interdum et malesuada fames ac ante ipsum primis in faucibus. Ut leo diam, viverra ultrices neque lobortis, eleifend viverra erat. 
\vspace{\onelineskip}
   \noindent 
   \textbf{Key-words}: keyword1, keyword2.
\end{otherlanguage*}
\end{resumo}
%falta uma pagina em branco
% ---
% inserir lista de ilustrações
% ---
\pdfbookmark[0]{\listfigurename}{lof}
\listoffigures*
\cleardoublepage
% ---

% ---
% inserir lista de quadros
% ---
%\pdfbookmark[0]{\listofquadrosname}{loq}
%\listofquadros*
%\cleardoublepage
% ---

% ---
% inserir lista de tabelas
% ---
\pdfbookmark[0]{\listtablename}{lot}
\listoftables*
\cleardoublepage
% ---

% ---
% inserir lista de algoritmos
% ---
%\listaalgoritmos
% ---

\begin{siglas}
    \item[ACID] Atomicidade, Consistência, Isolamento e Durabilidade (\textit{\foreignlanguage{english}{Atomicity, Consistency, Isolation, Durability}})
    \item[ANSI] Instituto Nacional Americano de Padrões (\textit{\foreignlanguage{english}{American National Standards Institute}})
\end{siglas}

\begin{simbolos}
%  \item[$ \circ $] Abertura morfológica
%  \item[$ \partial $] Derivada parcial
%  \item[$ \sigma $] Desvio Padrão
%  \item[$ \oplus $] Dilatação morfológica
%  \item[$ \ominus $] Erosão morfológica
%  \item[$ \bullet $] Fechamento morfológico
%  \item[$ \nabla $] Gradiente
%  \item[$ \mu $] Momento Regular Central (imagem)
%  \item[$ \eta $] Momento Regular Central Normalizado (imagem)
%  \item[$ \mathrm{e} $] Número de Euler
  \item[$ \delta $] Variação
\end{simbolos}


% ---
% Sumario
% ---
\pdfbookmark[0]{\contentsname}{toc}
\tableofcontents*
\cleardoublepage
% ---

% ----
% ELEMENTOS TEXTUAIS
% ----
\textual

\chapter{INTRODUÇÃO}\label{ch:introducao}

Lorem ipsum dolor sit amet, consectetur adipiscing elit. Phasellus bibendum erat vel dapibus finibus. Integer dignissim feugiat dui eget semper. Integer in lacinia mi. Nunc pellentesque accumsan ante quis egestas. Proin accumsan laoreet pellentesque. Nunc varius, orci quis maximus faucibus, nunc eros finibus lorem, vel aliquam lacus dolor ut lorem. Donec a justo in leo pretium pellentesque. Praesent a efficitur nunc. Donec pulvinar diam odio, ac congue magna bibendum vehicula. Nunc sed erat non est pulvinar sollicitudin mattis eget sem. Nulla laoreet egestas mattis. Donec eu fermentum mi. Nam dolor metus, convallis vel ligula eu, hendrerit scelerisque enim. Nunc quis dui id eros ultricies scelerisque ac nec neque.

Maecenas vestibulum ligula eu nibh ornare, quis placerat massa iaculis. In et nunc malesuada, elementum justo vitae, iaculis urna. Cras pharetra metus et velit aliquam, porttitor pharetra turpis varius. Lorem ipsum dolor sit amet, consectetur adipiscing elit. Duis consectetur tincidunt elementum. Interdum et malesuada fames ac ante ipsum primis in faucibus. Ut leo diam, viverra ultrices neque lobortis, eleifend viverra erat. Praesent imperdiet a diam varius pretium. Maecenas eget velit at metus porttitor faucibus. Phasellus feugiat dui eget elit suscipit volutpat. Quisque velit felis, suscipit quis leo a, accumsan laoreet metus. Quisque lacinia, arcu at dignissim molestie, metus ex consectetur libero, et efficitur erat leo ac purus. Vivamus lectus nisl, semper at dolor non, faucibus interdum leo. Aliquam auctor mi at massa vestibulum semper. Donec bibendum pretium tempus \cite{andrienko2012visual}.
\chapter{CONCEITOS}\label{ch:conceitos}

Lorem ipsum dolor sit amet, consectetur adipiscing elit. Phasellus bibendum erat vel dapibus finibus. Integer dignissim feugiat dui eget semper. Integer in lacinia mi. Nunc pellentesque accumsan ante quis egestas. Proin accumsan laoreet pellentesque. Nunc varius, orci quis maximus faucibus, nunc eros finibus lorem, vel aliquam lacus dolor ut lorem. Donec a justo in leo pretium pellentesque. Praesent a efficitur nunc. Donec pulvinar diam odio, ac congue magna bibendum vehicula. Nunc sed erat non est pulvinar sollicitudin mattis eget sem. Nulla laoreet egestas mattis. Donec eu fermentum mi. Nam dolor metus, convallis vel ligula eu, hendrerit scelerisque enim. Nunc quis dui id eros ultricies scelerisque ac nec neque.

Maecenas vestibulum ligula eu nibh ornare, quis placerat massa iaculis. In et nunc malesuada, elementum justo vitae, iaculis urna. Cras pharetra metus et velit aliquam, porttitor pharetra turpis varius. Lorem ipsum dolor sit amet, consectetur adipiscing elit. Duis consectetur tincidunt elementum. Interdum et malesuada fames ac ante ipsum primis in faucibus. Ut leo diam, viverra ultrices neque lobortis, eleifend viverra erat. Praesent imperdiet a diam varius pretium. Maecenas eget velit at metus porttitor faucibus. Phasellus feugiat dui eget elit suscipit volutpat. Quisque velit felis, suscipit quis leo a, accumsan laoreet metus. Quisque lacinia, arcu at dignissim molestie, metus ex consectetur libero, et efficitur erat leo ac purus. Vivamus lectus nisl, semper at dolor non, faucibus interdum leo. Aliquam auctor mi at massa vestibulum semper. Donec bibendum pretium tempus.

Suspendisse nec lacus pharetra, condimentum enim sit amet, aliquet ligula. Quisque ut sodales sapien, at ornare enim. Vivamus et feugiat neque. Maecenas sed libero vel magna aliquet mattis nec id tellus. Ut sem neque, commodo et cursus vitae, porta pretium nisi. Morbi non auctor tortor, id pharetra magna. Curabitur laoreet gravida ante, eu tincidunt justo. Phasellus rhoncus nulla fringilla orci ullamcorper, vel euismod augue dapibus. Morbi faucibus lacus dui, eget condimentum ligula consequat ut. Cum sociis natoque penatibus et magnis dis parturient montes, nascetur ridiculus mus.

Mauris non pulvinar mi, a tincidunt enim. Etiam rutrum lobortis nulla, id consectetur nibh venenatis eu. Class aptent taciti sociosqu ad litora torquent per conubia nostra, per inceptos himenaeos. Donec fringilla nunc ut orci euismod, id lacinia quam feugiat. Sed ultrices erat congue, convallis sem ut, tempus nisl. Etiam hendrerit bibendum velit, at faucibus dui hendrerit a. Donec tincidunt nisi convallis mi malesuada ultricies. Cras vitae auctor velit. Mauris quis tellus nunc. Pellentesque a sem quis leo pellentesque tincidunt quis sed libero. Etiam mattis tempor porttitor. Ut ac ipsum sit amet augue vehicula blandit id sed quam.


\chapter{TRABALHOS RELACIONADOS}\label{ch:trabalhorelacionado}

Lorem ipsum dolor sit amet, consectetur adipiscing elit. Phasellus bibendum erat vel dapibus finibus. Integer dignissim feugiat dui eget semper. Integer in lacinia mi. Nunc pellentesque accumsan ante quis egestas. Proin accumsan laoreet pellentesque. Nunc varius, orci quis maximus faucibus, nunc eros finibus lorem, vel aliquam lacus dolor ut lorem. Donec a justo in leo pretium pellentesque. Praesent a efficitur nunc. Donec pulvinar diam odio, ac congue magna bibendum vehicula. Nunc sed erat non est pulvinar sollicitudin mattis eget sem. Nulla laoreet egestas mattis. Donec eu fermentum mi. Nam dolor metus, convallis vel ligula eu, hendrerit scelerisque enim. Nunc quis dui id eros ultricies scelerisque ac nec neque.

Maecenas vestibulum ligula eu nibh ornare, quis placerat massa iaculis. In et nunc malesuada, elementum justo vitae, iaculis urna. Cras pharetra metus et velit aliquam, porttitor pharetra turpis varius. Lorem ipsum dolor sit amet, consectetur adipiscing elit. Duis consectetur tincidunt elementum. Interdum et malesuada fames ac ante ipsum primis in faucibus. Ut leo diam, viverra ultrices neque lobortis, eleifend viverra erat. Praesent imperdiet a diam varius pretium. Maecenas eget velit at metus porttitor faucibus. Phasellus feugiat dui eget elit suscipit volutpat. Quisque velit felis, suscipit quis leo a, accumsan laoreet metus. Quisque lacinia, arcu at dignissim molestie, metus ex consectetur libero, et efficitur erat leo ac purus. Vivamus lectus nisl, semper at dolor non, faucibus interdum leo. Aliquam auctor mi at massa vestibulum semper. Donec bibendum pretium tempus
\chapter{DESENVOLVIMENTO DA SOLUÇÃO}\label{ch:proposta}
Lorem ipsum dolor sit amet, consectetur adipiscing elit. Phasellus bibendum erat vel dapibus finibus. Integer dignissim feugiat dui eget semper. Integer in lacinia mi. Nunc pellentesque accumsan ante quis egestas. Proin accumsan laoreet pellentesque. Nunc varius, orci quis maximus faucibus, nunc eros finibus lorem, vel aliquam lacus dolor ut lorem. Donec a justo in leo pretium pellentesque. Praesent a efficitur nunc. Donec pulvinar diam odio, ac congue magna bibendum vehicula. Nunc sed erat non est pulvinar sollicitudin mattis eget sem. Nulla laoreet egestas mattis. Donec eu fermentum mi. Nam dolor metus, convallis vel ligula eu, hendrerit scelerisque enim. Nunc quis dui id eros ultricies scelerisque ac nec neque.

Maecenas vestibulum ligula eu nibh ornare, quis placerat massa iaculis. In et nunc malesuada, elementum justo vitae, iaculis urna. Cras pharetra metus et velit aliquam, porttitor pharetra turpis varius. Lorem ipsum dolor sit amet, consectetur adipiscing elit. Duis consectetur tincidunt elementum. Interdum et malesuada fames ac ante ipsum primis in faucibus. Ut leo diam, viverra ultrices neque lobortis, eleifend viverra erat. Praesent imperdiet a diam varius pretium. Maecenas eget velit at metus porttitor faucibus. Phasellus feugiat dui eget elit suscipit volutpat. Quisque velit felis, suscipit quis leo a, accumsan laoreet metus. Quisque lacinia, arcu at dignissim molestie, metus ex consectetur libero, et efficitur erat leo ac purus. Vivamus lectus nisl, semper at dolor non, faucibus interdum leo. Aliquam auctor mi at massa vestibulum semper. Donec bibendum pretium tempus
%____________________________________________________________________
\chapter{EXPERIMENTOS, RESULTADOS E DISCUSSÕES}\label{ch:experimentos}

Lorem ipsum dolor sit amet, consectetur adipiscing elit. Phasellus bibendum erat vel dapibus finibus. Integer dignissim feugiat dui eget semper. Integer in lacinia mi. Nunc pellentesque accumsan ante quis egestas. Proin accumsan laoreet pellentesque. Nunc varius, orci quis maximus faucibus, nunc eros finibus lorem, vel aliquam lacus dolor ut lorem. Donec a justo in leo pretium pellentesque. Praesent a efficitur nunc. Donec pulvinar diam odio, ac congue magna bibendum vehicula. Nunc sed erat non est pulvinar sollicitudin mattis eget sem. Nulla laoreet egestas mattis. Donec eu fermentum mi. Nam dolor metus, convallis vel ligula eu, hendrerit scelerisque enim. Nunc quis dui id eros ultricies scelerisque ac nec neque.

Maecenas vestibulum ligula eu nibh ornare, quis placerat massa iaculis. In et nunc malesuada, elementum justo vitae, iaculis urna. Cras pharetra metus et velit aliquam, porttitor pharetra turpis varius. Lorem ipsum dolor sit amet, consectetur adipiscing elit. Duis consectetur tincidunt elementum. Interdum et malesuada fames ac ante ipsum primis in faucibus. Ut leo diam, viverra ultrices neque lobortis, eleifend viverra erat. Praesent imperdiet a diam varius pretium. Maecenas eget velit at metus porttitor faucibus. Phasellus feugiat dui eget elit suscipit volutpat. Quisque velit felis, suscipit quis leo a, accumsan laoreet metus. Quisque lacinia, arcu at dignissim molestie, metus ex consectetur libero, et efficitur erat leo ac purus. Vivamus lectus nisl, semper at dolor non, faucibus interdum leo. Aliquam auctor mi at massa vestibulum semper. Donec bibendum pretium tempus
%____________________________________________________________________
\chapter{CONSIDERAÇÕES FINAIS}\label{ch:conclusoes}

Neste capítulo são apresentadas as conclusões, contribuições e propostas de trabalhos futuros.

\section{Conclusões}\label{sec:conclusoes}


% -----
% Referências bibliográficas
% -----
\bibliography{X.Referencias}

% Elementos pós-textuais
\begin{apendicesenv}

\chapter{Algoritmo XPTO}\label{ch:algoritmo_xpto}

Lorem ipsum dolor sit amet, consectetur adipiscing elit. Phasellus bibendum erat vel dapibus finibus. Integer dignissim feugiat dui eget semper. Integer in lacinia mi. Nunc pellentesque accumsan ante quis egestas. Proin accumsan laoreet pellentesque. Nunc varius, orci quis maximus faucibus, nunc eros finibus lorem, vel aliquam lacus dolor ut lorem. Donec a justo in leo pretium pellentesque. Praesent a efficitur nunc. Donec pulvinar diam odio, ac congue magna bibendum vehicula. Nunc sed erat non est pulvinar sollicitudin mattis eget sem. Nulla laoreet egestas mattis. Donec eu fermentum mi. Nam dolor metus, convallis vel ligula eu, hendrerit scelerisque enim. Nunc quis dui id eros ultricies scelerisque ac nec neque.

\chapter{Algoritmo XYZ}\label{ch:algoritmo_xyz}

Lorem ipsum dolor sit amet, consectetur adipiscing elit. Phasellus bibendum erat vel dapibus finibus. Integer dignissim feugiat dui eget semper. Integer in lacinia mi. Nunc pellentesque accumsan ante quis egestas. Proin accumsan laoreet pellentesque. Nunc varius, orci quis maximus faucibus, nunc eros finibus lorem, vel aliquam lacus dolor ut lorem. Donec a justo in leo pretium pellentesque. Praesent a efficitur nunc. Donec pulvinar diam odio, ac congue magna bibendum vehicula. Nunc sed erat non est pulvinar sollicitudin mattis eget sem. Nulla laoreet egestas mattis. Donec eu fermentum mi. Nam dolor metus, convallis vel ligula eu, hendrerit scelerisque enim. Nunc quis dui id eros ultricies scelerisque ac nec neque.

\end{apendicesenv}


\phantompart
\printindex

\end{document}
